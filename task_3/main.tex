\documentclass[12pt, letterpaper]{article}
\usepackage{hyperref}
\usepackage[warn]{mathtext}
\usepackage[utf8]{inputenc}
\usepackage[T2A]{fontenc}
\usepackage[russian]{babel}
\usepackage{amssymb, amsmath, multicol}
\usepackage{graphicx}
\usepackage[shortcuts,cyremdash]{extdash}
\usepackage{wrapfig}
\usepackage{floatflt}
\usepackage{lipsum}
\usepackage{verbatim}
\usepackage{concmath}
\usepackage{euler}
\usepackage{xcolor}
\usepackage{etoolbox}
\usepackage{fancyhdr}
\usepackage{subfiles}
\usepackage{enumitem}
\usepackage{amsthm}
\usepackage{indentfirst}
\usepackage{import}

\begin{document}

% НАЧАЛО ТИТУЛЬНОГО ЛИСТА
\begin{center}
    {\small ФЕДЕРАЛЬНОЕ ГОСУДАРСТВЕННОЕ АВТОНОМНОЕ ОБРАЗОВАТЕЛЬНОЕ\\ УЧРЕЖДЕНИЕ ВЫСШЕГО ОБРАЗОВАНИЯ\\ МОСКОВСКИЙ ФИЗИКО-ТЕХНИЧЕСКИЙ ИНСТИТУТ\\ (НАЦИОНАЛЬНЫЙ ИССЛЕДОВАТЕЛЬСКИЙ УНИВЕРСИТЕТ)\\ ФИЗТЕХ-ШКОЛА РАДИОТЕХНИКИ И КИБЕРНЕТИКИ}\\
    \hfill \break
    \hfill \break
    \hfill \break
    \Huge{Homework 3: Q/A}\\
  \end{center}

  \hfill \break
  \hfill \break
  \hfill \break
  \hfill \break
  \hfill \break
  \hfill \break

  \begin{flushright}
    \normalsize{Работу выполнил:}\\
    \normalsize{\textbf{Шурыгин Антон Алексеевич, группа М01-306}}\\
  \end{flushright}

  \begin{center}
    \normalsize{\textbf{Долгопрудный, 2024}}
  \end{center}


  \thispagestyle{empty} % выключаем отображение номера для этой страницы

  % КОНЕЦ ТИТУЛЬНОГО ЛИСТА

  \newpage
  \thispagestyle{plain}
  \tableofcontents
  \thispagestyle{plain}
  \newpage



\section{Вопрос}

\textbf{Q: } Объясните смысл и напишите формулы для следующих метрик про
изводительности процессора IPC, CPI, Performance, Dynamic Power

\textbf{A: }

\begin{enumerate}
    \item IPC -- среднее число инструкций за такт.
    \item CPI -- cреднее число тактов на выполнение одной инструкции.
    \item Performance -- скорость исполнения инструкций.
    \item Dynamic power -- затраты энергии на перезарядку паразитных емкостей при переключении транзистров в процессоре.
\end{enumerate}

\begin{equation}
    IPC = \frac{instructions \:\: count}{clock \:\: cycles}
\end{equation}

\begin{equation}
    CPI = \frac{1}{IPC}
\end{equation}

\begin{equation}
    Perfomance = \frac{1}{Time} = \frac{1}{N_{instrs} \cdot CPI \cdot T_{cycle}} = \frac{1}{N_{instrs}} \cdot IPC \ cdot f
\end{equation}

\begin{equation}
    Dynamic Power = \frac{C_{eff} \cdot (V_{dd})^{2} \cdot f \cdot a}{2}
\end{equation}
, где $a$ -- частота переключения транзистора, $f$ -- тактовая частота.
\newpage

\section{Вопрос}

\textbf{Q: } Что такое суперскалярный (superscalar) процессор?

\textbf{A: } Суперскалярный (superscalar) процессор - процессор, способный выпонлять две и более инструкций за такт за счет нескольких одинаковых вычислительных устройств, незаивисимо обрабатывающих инструкции одного потока.

\newpage

\section{Вопрос}

\textbf{Q: } Какие типы зависимостей по данным существуют? Приведите приме
ры аппаратных оптимизаций, которые позволяют сократить связан
ные задержки или разрешить каждый тип зависимостей

\textbf{A: }


\newpage

\section{Вопрос}

\textbf{Q: } С какой целью инструкцию Store разделяют на микро-операции STA
(Store address calculation) и STD (Store data calculation) ?

\textbf{A: }

\begin{enumerate}
    \item STA -- микрооперация, позволяющая вычислить адрес STORE инструкции
    \item STD -- микрооперация, позволяющая вычислить данные STORE инструкции
\end{enumerate}

В идеальном процессе исполнения програаммы зависимости между STORE и LOAD минимальны. Откуда вообще берутся зависимости? Наприимер, зависимость между STORE и LOAD
существуют, если адреса обращения в память инструкций пересекаются. В простом случае исполнения STORE адрес, по которому происходит загрузка в память, не известен до тех пор, пока не будут готовы данные для записи.
Микрооперация STA позволяет вычилсить адрес обращения в память, пока еще не вычислены данные. Таким образом, полуцчается уменьшить число STORE и LOAD зависимостей.

\newpage

\section{Вопрос}

\textbf{Q: } Объясните назначение и функции следующих аппаратных структур:
ROB, Scheduler Queue (Issue Queue, Reservation Station), RAT, PRF, Load Buffer, Store Buffer

\textbf{A: }

\begin{enumerate}
    \item ROB, re-order buffer -- буфер для хранения последовательного порядка инструкции при Out Of Order исполнении. При OOO инструкции исполняются непоследовательно, однако порядок исполнения важен и нужен, чтобы знать архитектурное состояние. Таким обзазом, re-order buffer обеспечивает спекулятивное инструкций. Инструкция вытесняется из ROB, обновляя архитектурное состяоне, если она была исполнена и последняя в буфере.
    \item Scheduler Queue --
    \item RAT -- таблица соотвествия физических и логических регистров, используемых при алгоритме переименования регистров.
    \item PRF -- Physical Register File, регистровый файл, в котором описаны все физические регистры процессора.
    \item Load Buffer, Store Buffer -- буферы для хранения STORE и LOAD инструкций, которые обеспечивают разрешение заивисмостей по данным между STORE и LOAD.
\end{enumerate}

\newpage

\section{Вопрос}

\textbf{Q: } Пусть каждая 5-ая инструкция в процессоре это Branch. Предсказатель
переходов имеет точность 90\%. Оцените, ROB какого максимального
размера имеет смысл для такого процессора.

\textbf{A: }

\newpage

\section{Вопрос}

\textbf{Q: }  Что такое Memory Disambiguation?

\textbf{A: }

\newpage

\section{Вопрос}

\textbf{Q: } В чем заключается проблема со спекулятивным исполнением Store
инструкций?

\textbf{A: }

\newpage

\section{Вопрос}

\textbf{Q: } Что такое Store forwarding и Load speculation в OOO процессоре?

\textbf{A: }

\newpage

\section{Вопрос}

\textbf{Q: } Что такое Simultaneous Multithreading?

\textbf{A: }

\newpage



\end{document}
